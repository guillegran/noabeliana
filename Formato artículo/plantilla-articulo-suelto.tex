\documentclass[autocontact]{gaceta}
% (* o \documentclass{gaceta} si el art\'{\i}culo no tiene firma *)

%---------------------
%%%%%% NOTA T\'ECNICA (ign\'orala si no la entiendes) %%%%%%
% Si, en lugar de pdflatex (opci\'on por defecto), vas a usar 
% latex, puedes reproducir el tama\~no final del papel con el 
% que se imprimir\'a la Gaceta a\~nadiendo la opci\'on dvips:
% \documentclass[dvips]{gaceta}
%---------------------

%---------------------
% No olvides descargar el archivo gaceta.cls y ponerlo 
% en la misma carpeta que este documento, o donde LaTeX 
% lo sepa encontrar.
% Recuerda que La Gaceta prefiere art\'{\i}culos que 
% no superen las 20 p\'aginas.
%---------------------

%---------------------
% La Gaceta se har\'a con esto activado, 
% pero no hace falta ahora si no quieres:
%\usepackage[T1]{fontenc}
%\usepackage{lmodern}

% Puedes usar esto (u otra codificaci\'on) si quieres:
\usepackage[utf8]{inputenc}

\usepackage[spanish]{babel}
%---------------------

%---------------------
% Esto ya lo carga el estilo de La Gaceta:
%\usepackage{amsmath, amsthm, amssymb}
%\usepackage{url} % <-- Para p\'aginas web o similar: \url{...}
%\usepackage{graphicx}
%---------------------

%---------------------
% Carga esto para gr\'aficos si lo necesitas:
% \usepackage{wrapfig}
%---------------------

%---------------------
% <<< ESTO SE AJUSTAR\'A AL EDITAR CADA N\'UMERO DE LA GACETA >>>
\setcounter{page}{1} 
\journame{La Gaceta de la RSME}
\yearofpublication{0000}
\volume{00}
\issuenumber{0}
%---------------------

%---------------------
% Deja esto as\'{\i}:
\belongstopart{Art\'{\i}culos} 
%\belongstopart{Secciones} 
%\belongstopart{Noticias de la Sociedad} 
%\belongstopart{Actualidad} 
%\nopart
%---------------------

%---------------------
%Mis entornos y paquetes:
\usepackage{tikz,tikz-cd}
\usetikzlibrary{arrows, babel}
\usepackage[scr=boondox]{mathalfa}

\newtheorem{thm}{Teorema}[section]
\newtheorem{prop}[thm]{Proposición}
\newtheorem{lema}{Lema}
\newtheorem*{lema*}{Lema}
\newtheorem{corol}[thm]{Corolario}
\theoremstyle{definition} \newtheorem{defn}[thm]{Definición}
\theoremstyle{definition} \newtheorem{ejemplo}[thm]{Ejemplo}
\theoremstyle{definition} \newtheorem{ejercicio}[thm]{Ejercicio}
\newtheorem{obs}[thm]{Observación}


\def\CC{\mathbb{C}}
\def\RR{\mathbb{R}}
\def\OO{\mathscr{O}}
\def\FF{\mathscr{F}}
\def\GG{\mathscr{G}}
\def\HH{\mathscr{H}}
\def\Op{\mathbf{Op}}
\def\UU{\mathfrak{U}}
\DeclareMathOperator{\id}{id}
\DeclareMathOperator{\Span}{span}
\DeclareMathOperator{\delbar}{\bar{\partial}}
\DeclareMathOperator{\coker}{coker}
\DeclareMathOperator{\im}{im}
\DeclareMathOperator{\pr}{pr}



%---------------------
\title{Introducción a la teoría de Hodge no abeliana}
\author{Guillermo Gallego Sánchez}
%---------------------

%---------------------
\contact{Un autor, Dpto. de Matem\'aticas, Universidad de \dots}
{autor@uni.es}{http://www.uni.es/personales/autor.html}
%% SINTAXIS (\'usense tantos de estos como autores haya):
%\contact{nombre y direcci\'on autor 1}{email autor 1}{p\'agina web autor 1}
%\contact{nombre y direcci\'on autor 2}{email autor 2}{p\'agina web autor 2}
% (IMPORTANTE: se puede dejar vac\'{\i}o lo que se quiera)
%---------------------

%---------------------------
\begin{document}
%---------------------------

%---------------------
\maketitle
%---------------------

\begin{abstract}
Esto es un resumen del art\'{\i}culo. Hay que incluirlo en todos los art\'{\i}culos para la secci\'on general.
\end{abstract}


%---------------------
%%% Se puede poner si se quiere:
%\section*{Introducci\'on} % (o similar)
%---------------------


\section{Haces y cohomología}
\subsection{Prehaces y haces}
Supongamos que $S$ es un espacio topológico. Asociada a $S$ podemos considerar una categoría, que denotamos por $\Op_S$, cuyos objetos son los subconjuntos abiertos de $S$ y cuyos morfismos están dados por las inclusiones entre abiertos $V\subset U$.

\begin{defn}
  Un \emph{prehaz de grupos} (o, simplemente, un \emph{prehaz}) en $S$ es un functor contravariante $\Op_S \rightarrow \mathbf{Grp}$.
\end{defn}

En otras palabras, un prehaz $\FF$ asigna, a cada subconjunto $U\subset S$, un grupo $\FF(U)$, cuyos elementos llamamos \emph{secciones de $\FF$ en $U$}, y, a cada inclusión $V\subset U$, un \emph{homomorfismo de restricción} $$r^U_V:\FF(U) \rightarrow \FF(V).$$

\begin{ejemplo}
El ejemplo más importante de prehaz es el \emph{prehaz de las funciones continuas}. Si $S$ es un espacio topológico y $G$ es un grupo cualquiera, definimos el prehaz $C_S(-,G)$ en $S$ dado por
\begin{equation*}
  C_S(U,G) = \left\{ \text{Aplicaciones continuas } U\rightarrow G \right\},
\end{equation*}
y con homomorfismos de restricción dados, precisamente, por las restricciones
\begin{align*}
  r^U_V :C_S(U,G) &\longrightarrow C_S(V,G)\\ 
    f &\longmapsto f|_V. 
  \end{align*}
\end{ejemplo}

  Tomando este ejemplo como paradigma, en general vamos a adoptar la notación $f|_V= r^U_V(f)$ para cualquier haz $\FF$ y cualquier $f\in \FF(U)$.

  El prehaz de las funciones continuas tiene una serie de propiedades «buenas» extra que nos gustaría tener en general. La primera propiedad se conoce como \emph{localidad} y consiste en que, si tomamos $U\subset S$ un subconjunto abierto y $\mathfrak{U}$ un recubrimiento por abiertos de $U$, esto es $U=\bigcup_{V \in \mathfrak{U}} V$, entonces, si $f,g:U\rightarrow G$ son dos funciones tales que $f|_{V} = g|_{V}$, para cada $V\in \mathfrak{U}$, tenemos que $f=g$. La segunda propiedad es la propiedad de \emph{pegado}: si para cada $V\in \mathfrak{U}$ podemos definir una $f_V:V\rightarrow G$ de tal forma que $f_V|_{V\cap W} = f_W|_{V\cap W}$ para cada $V,W \in \mathfrak{U}$, entonces existe una función $f:U\rightarrow G$ tal que $f|_V=f_V$ para cada $V\in \mathfrak{U}$. Estas dos propiedades pueden juntarse en una sola si añadimos a la propiedad de pegado la condición extra de que la $f$ obtenida sea única. Así, definimos:

  \begin{defn}
    Un prehaz $\FF$ es un \emph{haz} si, para cada subconjunto abierto $U\subset S$, dado un recubrimiento por abiertos $\mathfrak{U}$ de $U$ y una colección de elementos $f_V \in \FF(V)$ para cada $V\in \mathfrak{U}$ tal que, para cada $V,W\in \mathfrak{U}$,
    \begin{equation*}
      f_V|_{V\cap W} = f_W|_{V\cap W},
    \end{equation*}
    entonces existe una única $f\in \FF(U)$ tal que $f|_V=f_V$ para cada $V \in \mathfrak{U}$.
  \end{defn}

  Veamos ahora la condición de ser un haz de una forma distinta. Consideremos un prehaz $\FF$, un subconjunto abierto $U\subset S$ y un recubrimiento por abiertos $\UU$ de $U$. Además, será conveniente que fijemos un buen orden en el conjunto $\UU$. 

  Definimos el conjunto de las \emph{$0$-cocadenas de \v{C}ech} de $\FF$ en $\UU$ como
  \begin{equation*}
    C^0(\UU,\FF) = \prod_{V\in \UU} \FF(V).
  \end{equation*}
  Definimos también el conjunto de las \emph{$1$-cocadenas de \v{C}ech} de $\FF$ en $\UU$ como
  \begin{equation*}
    C^1(\UU,\FF) = \prod_{V<W\in \UU} \FF(V\cap W).
  \end{equation*}
  Los elementos de $C^0(\UU,\FF)$ pueden escribirse como $(f_V)_{V\in \UU}$ y los elementos de $C^1(\UU,\FF)$ como $(f_{VW})_{V<W \in \UU}$. Nótese que podemos olvidarnos del buen orden fijado en $\UU$ si simplemente, para $V<W$, definimos $f_{WV}=f_{VW}^{-1}$. 
  
  Ahora, las aplicaciones de restricción inducen aplicaciones naturales
  \begin{align*}
    d_0 :\FF(U)&\longrightarrow C^0(\UU,\FF)\\ 
    f &\longmapsto (f|_V)_{V\in \UU}
    \end{align*}
    \begin{align*}
      d_1' :C^0(\UU,\FF)&\longrightarrow C^1(\UU,\FF)\\ 
      (f_V)_{V\in \UU} &\longmapsto (f_V|_{V\cap W})_{V,W \in \UU}
      \end{align*}
      y
    \begin{align*}
      d_1'' :C^0(\UU,\FF)&\longrightarrow C^1(\UU,\FF)\\ 
      (f_V)_{V\in \UU} &\longmapsto (f_W|_{V\cap W})_{V,W \in \UU}.
      \end{align*}

  Podemos dar entonces una nueva formulación de la condición de haz. Un prehaz $\FF$ en $S$ es un haz si, para cada subconjunto abierto $U\subset S$ y para cualquier recubrimiento por abiertos $\UU$ de $U$, la aplicación $d_0$ es inyectiva y el diagrama
  \begin{center}
    \begin{tikzcd}
      \FF(U) \arrow[hook]{r}{d_0} & C^0(\UU,\FF) \arrow[shift left=.75ex]{r}{d_1'} \arrow[shift right=.75ex, swap]{r}{d_1''} & C^1(\UU,\FF),
    \end{tikzcd}
  \end{center}
  es un ecualizador. Esto quiere decir que
  \begin{equation*}
    \FF(U)=\left\{ f\in C^0(\UU,\FF) : d_1'(f) = d_1''(f) \right\}.
  \end{equation*}
  Más aún, nótese que al estar definidos como productos cartesianos, los conjuntos $C^0(\UU,\FF)$ y $C^1(\UU,\FF)$ heredan una estructura de grupo, y podemos definir la aplicación
  \begin{align*}
    d_1 :C^0(\UU,\FF)&\longrightarrow C^1(\UU,\FF)\\ 
    (f_V)_{V\in \UU} &\longmapsto (f_V|_{V\cap W} f_W|_{V\cap W}^{-1})_{V,W \in \UU}. 
    \end{align*}
    Así, la condición de haz equivale a decir que $\FF(U)= \ker d_1$, o lo que es lo mismo, que la siguiente sucesión es exacta
    \begin{center}
      \begin{tikzcd}
	1 \rar & \FF(U) \arrow{r}{d_0} & C^0(\UU,\FF) \arrow{r}{d_1} & C^1(\UU,\FF).
      \end{tikzcd}
    \end{center}

    \subsection{Sucesiones exactas y cohomología de haces}
    Comenzamos este apartado con varias definiciones importantes.
    \begin{defn}
      Un \emph{morfismo de haces} $\FF \rightarrow \GG$ en un espacio topológico $S$ es una transformación natural entre los prehaces subyacentes. Es decir, es una colección de aplicaciones $\varphi_U:\FF(U) \rightarrow \GG(U)$ para cada subconjunto $U\subset S$ tal que, para cada inclusión $V\subset U$, el siguiente diagrama conmuta
      \begin{center}
	\begin{tikzcd}
\FF(U)	  \arrow{rr}{\varphi_U}\arrow{dd}[anchor=east]{r^U_V} && \GG(U) \arrow{dd}[anchor=west]{r^U_V} \\ 
	   && \\
	   \FF(V) \arrow{rr}[anchor=south]{\varphi_V } && \GG(V).
	 \end{tikzcd}
       \end{center}
    \end{defn}
    \begin{defn}
      Sea $S$ un espacio topológico y $\FF$ un haz en $S$. Definimos la \emph{espiga} de $\FF$ en un punto $p \in S$ como el límite directo
      \begin{equation*}
	\FF_p = \varinjlim_{U \ni p} \FF(U).
      \end{equation*}
    \end{defn}
    En otras palabras, $\FF_p$ es el conjunto de clases de equivalencia de pares $(U,f)$, con $U$ un entorno de $p$ y $f\in \FF(U)$, donde la relación de equivalencia está dada por:
    \begin{center}
      $(U,f) \sim (V,g)$ si existe $W\ni p$ tal que $W\subset U\cap V$ y $f|_W = g|_W$.
    \end{center}

    \begin{defn}
      Sean $\FF$, $\GG$ y $\HH$ haces en un espacio topológico $S$. Decimos que una sucesión de morfismos de haces
      \begin{center}
	\begin{tikzcd}
	  \FF \rar & \GG \rar & \HH
	\end{tikzcd}
      \end{center}
      es \emph{exacta} en $\GG$ si, para todo $p\in S$, la correspondiente sucesión inducida en las espigas
      \begin{center}
	\begin{tikzcd}
	  \FF_p \rar & \GG_p \rar & \HH_p
	\end{tikzcd}
      \end{center}
      es exacta en $\GG_p$; esto es, $\ker(\GG_p \rightarrow \HH_p) = \im(\FF_p \rightarrow \GG_p)$.

    \end{defn}

      Nótese que la exactitud de una sucesión de haces ha sido definida al nivel de las espigas y uno en principio no debería esperar que la sucesión correspondiente
      \begin{center}
	\begin{tikzcd}
	  \FF(S) \rar & \GG(S) \rar & \HH(S)
	\end{tikzcd}
      \end{center}
      sea exacta. El siguiente ejemplo aclara que, en efecto, la exactitud no tiene por qué preservarse al tomar secciones globales.

    \begin{ejemplo}
      Consideremos $S=\CC\setminus \{0\}$ el plano complejo «agujereado». Sobre este espacio topológico consideramos los siguientes haces:
      \begin{itemize}
	\item $C^\infty_S$, es el \emph{haz de funciones diferenciables} en $S$ con valores en $\CC$, definido de forma que $C^\infty_S(U)=C^\infty(U,\CC)$ es el grupo aditivo de las funciones diferenciables en $U$ con valores en $\CC$, para cada subconjunto abierto $U\subset S$.
	\item $(C^\infty_S)^*$, está definido de forma que $(C^\infty_S)^*(U)$ es el grupo multiplicativo de las funciones diferenciables \emph{que no se anulan} en $U$.
	\item $\underline{\mathbb{Z}}$ es el \emph{haz localmente constante con valores en $\mathbb{Z}$}, definido de forma que $\underline{\mathbb{Z}}(U)$ es el conjunto de aplicaciones localmente constantes $U\rightarrow \mathbb{Z}$.
      \end{itemize}
      La aplicación exponencial $\exp:C^\infty_S(U) \rightarrow (C^\infty_S)^*(U)$ induce una sucesión exacta de haces (nótese que ahora denotamos por $0$ al grupo trivial ya que, al estar trabajando con haces de grupos abelianos, usamos notación aditiva)
      \begin{center}
	\begin{tikzcd}
	  0 \rar & 2\pi i \underline{\mathbb{Z}} \rar & C^\infty_S \arrow{r}{\exp} & (C^\infty_S)^* \rar & 0,
	\end{tikzcd}
      \end{center}
      ya que, para todo punto $p\in S$, podemos tomar un entorno simplemente conexo $D\ni p$, donde el logaritmo está bien definido (salvo un factor $2\pi i k$, para $k\in \mathbb{Z}$). Sin embargo, globalmente en $S$ el logaritmo no está bien definido. Por ejemplo, si consideramos la función $f(z)=z$, tenemos que $f\in (C^\infty_S)^*(S)$, pero $f\not \in \exp(C^\infty_S(S))$. Esto prueba que la sucesión inducida en las secciones globales no es exacta.
    \end{ejemplo}
    
    En cualquier caso, tomar secciones globales sí que respeta la exactitud «por la izquierda»:

    \begin{prop}
      Una sucesión exacta corta de haces en un espacio topológico $S$,
      \begin{center}
	\begin{tikzcd}
	  1 \rar & \FF \arrow{r}{i} & \GG \arrow{r}{\pi} & \HH \rar & 1,
	\end{tikzcd}
      \end{center}
      induce una sucesión exacta
      \begin{center}
	\begin{tikzcd}
	  1 \rar & \FF(S) \arrow{r}{i_S} & \GG(S) \arrow{r}{\pi_S} & \HH(S).
	\end{tikzcd}
      \end{center}
    \end{prop}
    \begin{proof}
      Sea $f\in \FF(S)$ tal que $i_S(f)=1$. Para cada $p\in S$ tenemos que $i_p(f_p)=1$ y, como $i_p$ es inyectiva, $f_p=1$. Como esto es cierto para todo $p\in S$, tenemos que $f=1$. Esto prueba que $i_S$ es inyectiva, luego la sucesión es exacta en $\FF(S)$.

      Tomemos ahora $f\in \FF(S)$. Para cada $p \in S$ tenemos $\pi_p(i_p(f_p))=1$, lo que implica que $\pi_S(i_S(f_S))=1$, de modo que $\im i_S \subset \ker \pi_S$. Por otra parte, si $g\in \GG(S)$ es tal que $\pi_S(g)=1$ entonces, para todo $p\in S$, tenemos que $\pi_p(g_p)=1$. Por tanto, existe un $f_p \in \FF_p$ tal que $i_p(f_p)=g_p$. Esto implica que para cada $p\in S$ existen un entorno abierto $U\subset S$ de $p$ y un elemento $f_U\in \FF(U)$ tal que $i_U(f_U)=g_U$. Más aún, en una intersección $U\cap V$ tenemos
      \begin{equation*}
	i_{U\cap V}(f_U|_{U\cap V}) = g|_{U\cap V} = i_{U\cap V}(f_V|_{U\cap V})
      \end{equation*}
      y, como ya hemos probado que $i$ sigue siendo inyectiva al pasar a secciones globales, tenemos que $f_U|_{U\cap V} = f_V|_{U\cap V}$. Por tanto, podemos pegar las $f_U$ para obtener un elemento $f\in \FF(S)$ tal que $f|_U=f_U$ y que cumple $i_S(f)=g$. Esto demuestra que $\ker \pi_S \subset \im i_S$. En conclusión, hemos visto que la sucesión es también exacta en $\GG(S)$.
    \end{proof}

    Cabe ahora hacerse una pregunta natural: ¿qué es lo que falla en $\HH(S)$? ¿Por qué $\pi_S$ no es sobreyectiva? Es decir, buscamos cuál es la obstrucción que existe para que la exactitud pase de la situación local a la global. Lo que vamos a ver es que esta obstrucción está dada precisamente por la \emph{cohomología de haces}.

    Consideremos un elemento $h\in \HH(S)$. Entonces, como $\pi_p$ es sobreyectiva, para todo $p\in S$ existe algún $g_p \in \GG_p$ tal que $h_p=\pi(g_p)$. Esto implica que podemos tomar un recubrimiento por abiertos $\mathfrak{U}$ de $S$ y, para cada $U\in \mathfrak{U}$, un elemento $g_U\in \GG(U)$ tal que $h|_U=\pi_U(g_U)$. Si pudieramos pegar las $g_U$ para obtener una sección global $g\in \GG(X)$ tal que $g|_U = g_U$, entonces $h=\pi(g)$ y tendríamos que $\pi$ es sobreyectiva. Por tanto, la obstrucción para la sobreyectividad de $\pi$ viene dada precisamente por la condición de pegado, ya que en general $g_U|_{U\cap V} \neq g_V|_{U\cap V}$. 
    
    Tomemos entonces el elemento $g_Ug_V^{-1} \in \GG(U\cap V)$. Tenemos que
    \begin{equation*}
      \pi_{U\cap V} (g_U g_V^{-1}) = h h^{-1} = 1.
    \end{equation*}
    Por la exactitud en $\GG(U\cap V)$, existe un elemento $f_{UV}\in \FF(U\cap V)$ tal que
    \begin{equation*}
      i_{U\cap V} (f_{UV}) = g_U g_V^{-1}.
    \end{equation*}
    Reuniendo estos elementos, obtenemos una $1$-cocadena $(f_{UV})_{U,V \in \mathfrak{U}} \in C^1(\UU,\FF)$. Ahora, esta cocadena satisface una propiedad extra,
    \begin{equation*}
      i_{U\cap V\cap W}(f_{UV} f_{VW} f_{UW}^{-1}) = g_U g_V^{-1} g_V g_W^{-1} (g_U g_W^{-1})^{-1} = 1,
    \end{equation*}
    de modo que, como $i_{U\cap V\cap W}$ es inyectiva, la cocadena satisface la \emph{condición de cociclo}:
    \begin{equation*}
      f_{UV} f_{VW} = f_{UW}.
    \end{equation*}

    \begin{defn}
      Decimos que una $1$-cocadena de \v{C}ech $(f_{UV})_{U,V\in \mathfrak{U}}\in C^1(\UU,\FF)$ es un \emph{$1$-cociclo de \v{C}ech} de $\FF$ en $\UU$ si
    \begin{equation*}
      f_{UV} f_{VW} = f_{UW},
    \end{equation*}
    en $U\cap V \cap W$, para cada $U,V,W \in \UU$. El conjunto de los $1$-cociclos de \v{C}ech de $\FF$ en $\UU$ se denota por $Z^1(\FF,\UU)$.
    \end{defn}

    Ingenuamente, podríamos pensar que para estudiar la obstrucción a la exactitud $\HH(S)$ se trataría de definir una aplicación $\HH(S) \rightarrow Z^1(\UU,\FF)$, enviando cada $h\in \HH(S)$ al cociclo que acabamos de construir. Sin embargo, en seguida nos damos cuenta de que el recubrimiento $\UU$ depende del elemento $h$, de modo que esto no funciona. La forma de resolver este problema es considerando \emph{refinamientos}.

    \begin{defn}
      Sea $S$ un espacio topológico y $\UU$ y $\UU'$ dos recubrimientos por abiertos de $S$. Decimos que $\UU'$ es un \emph{refinamiento} de $\UU$ si existe una \emph{aplicación de refinamiento} $\tau:\UU' \rightarrow \UU$ tal que $U'\subset \tau(U')$ para cada $U'\in \UU$.
    \end{defn}

    Una aplicación de refinamiento induce una aplicación en las $1$-cocadenas
    \begin{align*}
      \tau^* :C^1(\UU,\FF)&\longrightarrow C^1(\UU',\FF) \\ 
      (f_{U V})_{U,V \in \UU} &\longmapsto (f_{\tau(U'),\tau(V')})_{U',V' \in \UU'}. 
      \end{align*}
      Más aún, claramente $\tau^*$ respeta la condición de cociclo, luego induce una aplicación
      \begin{align*}
	\tau^* :Z^1(\UU,\FF) &\longrightarrow Z^1(\UU',\FF). 
	\end{align*}
	Nótese además que esta aplicación depende de la elección de aplicación de refinamiento $\tau$. 
	
	Sin embargo, si $\eta:\UU'\rightarrow \UU$ es otra aplicación de refinamiento, aplicando la condición de cociclo tenemos
	\begin{equation*}
	  (\eta^* f)_{U'V'} = f_{\eta(U') \eta(V')} = f_{\eta(U') \tau(U')} f_{\tau(U') \eta(V')} =f_{\eta(U') \tau(U')} f_{\tau(U') \tau(V')}f_{\tau(V') \eta(V')}.
	\end{equation*}
	Si definimos la $0$-cocadena $$(f_{U'})_{U'\in \UU'}= (f_{\eta(U')\tau(U')})_{U'\in \UU} \in C^0(\UU',\FF),$$
	entonces
	\begin{equation*}
	  (\eta^* f)_{U'V'} = f_{U'} (\tau^* f)_{U'V'} f_{V'}^{-1}.
	\end{equation*}

	En general, para cualquier recubrimiento por abiertos $\UU$, el grupo de las $0$-cocadenas actúa sobre las $1$-cocadenas por «conjugación»:
	\begin{align*}
	   C^0(\UU,\FF) \times C^1(\UU, \FF) &\longrightarrow C^1(\UU, \FF) \\ 
	   ( (f_U)_{U\in \UU}, (f_{UV})_{U,V \in \UU} )  &\longmapsto ( f_U f_{UV} f_V^{-1} )_{U,V \in \UU}. 
	  \end{align*}
	  Esta acción respeta la condición de cociclo y por tanto define una acción de $C^0(\UU,\FF)$ en $Z^1(\UU,\FF)$.

	  \begin{defn}
	    El conjunto de órbitas de esta acción se denota por $H^1(\UU,\FF)$ y se llama el \emph{primer conjunto de cohomología de \v{C}ech} de $\FF$ en $\UU$.
	  \end{defn}

	  \begin{obs}
	    Supongamos que $\FF$ es un haz de grupos abelianos, de modo que pasamos a usar notación aditiva. En este caso, la aplicación $d_1:C^0(\UU,\FF) \rightarrow C^1(\UU,\FF)$ definida en el apartado anterior puede escribirse como 
	    \begin{equation*}
	      d_1 ( (f_U)_{U\in \UU} ) = ( f_U|_{U\cap V} - f_V|_{U\cap V} )_{U,V \in \UU},
	    \end{equation*}
	    mientras que la acción de $C^0(\UU,\FF)$ en $C^1(\UU,\FF)$ ahora se escribe como
	    \begin{equation*}
	      (f_U)_{U\in \UU} \cdot (f_{UV})_{U,V \in \UU} = (f_{UV} + f_U - f_V)_{U,V \in \UU} = (f_{UV})_{U,V \in \UU} + d_1((f_U)_{U\in \UU}).
	    \end{equation*}
	    Por tanto, la cohomología de \v{C}ech en este caso viene dada simplemente por el grupo cociente
	    \begin{equation*}
	      H^1(\UU,\FF) = \frac{Z^1(\UU,\FF)}{d_1(C^0(\UU,\FF))}.
	    \end{equation*}
	  \end{obs}
	
	  Por lo que hemos visto antes, es claro ahora que una aplicación de refinamiento $\tau:\UU'\rightarrow \UU$ induce una aplicación en cohomología
	  \begin{align*}
	    \tau^* :H^1(\UU,\FF)&\longrightarrow H^1(\UU',\FF)
	    \end{align*}
	    que \emph{no depende de la aplicación de refinamiento} $\tau$. Estas aplicaciones de refinamiento forman un sistema dirigido, de modo que podemos dar la siguiente definición.

	    \begin{defn}
	      Definimos el \emph{primer conjunto de cohomología de haces} de $\FF$ en $S$ como el límite directo 
	      \begin{equation*}
		H^1(S,\FF) = \varinjlim_{\UU} H^1(\UU,\FF).
	      \end{equation*}
	    \end{defn}

	    En otras palabras, consideramos el conjunto de pares $(\UU,f)$, con $\UU$ un recubrimiento por abiertos de $S$ y $f\in H^1(\UU,\FF)$ y cocientamos por la relación de equivalencia:
	    \begin{center}
	      $(\UU,f) \sim (\UU',f')$ si existen $\tau,\tau':\UU'' \rightarrow \UU, \UU'$ tales que $\tau^*(f) = \tau'^*(f')$.
	    \end{center}

	    Finalmente, dado un elemento $h\in \HH(S)$, si tomamos el recubrimiento por abiertos $\UU$ y el cociclo $(f_{UV})_{U,V \in \UU}$ construidos antes, tomamos su clase de cohomología de \v{C}ech y su clase de equivalencia en el límite directo, obtenemos una aplicación
	    \begin{align*}
	      \delta :\HH(S) &\longrightarrow H^1(S,\FF)\\ 
	      h &\longmapsto [(\UU,[(f_{UV})_{U,V \in \UU})] .
	      \end{align*}
	      El conjunto $H^1(S,\FF)$ no es un grupo, pero tiene un elemento distinguido $1$, la clase de equivalencia de cualquiera de los elementos neutros de los grupos de cocadenas. Tiene sentido entonces hablar de el núcleo de una aplicación con valores en $H^1(S,\FF)$ (la preimagen de $1$), de modo que podemos considerarlo como un elemento de una sucesión exacta. Lo que hemos visto con todo esto es que un elemento $h\in \HH(S)$ está en la imagen de $\pi_S$ precisamente si y solo si $\delta(h)=1$, lo que quiere decir que tenemos una sucesión exacta
	      \begin{center}
		\begin{tikzcd}
		  0 \rar & \FF(S) \arrow{r}{i_S} & \GG(S) \arrow{r}{\pi_S} & \HH(S) \arrow{r}{\delta} & H^1(S,\FF).
		\end{tikzcd}
	      \end{center}

	      Para concluir este apartado, cabe mencionar que la cohomología tiene un carácter functorial. Concretamente, cualquier morfismo de haces $\varphi:\FF \rightarrow \GG$ induce de forma natural una aplicación en las $1$-cocadenas
	      \begin{align*}
		\varphi_* :C^1(\UU,\FF) &\longrightarrow C^1(\UU,\GG)\\ 
		(f_{UV})_{U,V\in \UU} &\longmapsto (\varphi_{U\cap V}(f_{UV}) )_{U,V \in \UU}. 
		\end{align*}
		Es claro que esta aplicación desciende bien a las cohomologías, de modo que da una aplicación
		\begin{align*}
		   \varphi_*:H^1(S,\FF) &\longrightarrow H^1(S,\GG).
		  \end{align*}
		  Tenemos entonces el siguiente resultado:

		  \begin{thm}[La sucesión exacta larga en cohomología]
  Una sucesión exacta corta de haces en un espacio topológico $S$,
  \begin{center}
    \begin{tikzcd}
	  1 \rar & \FF \arrow{r}{i} & \GG \arrow{r}{\pi} & \HH \rar & 1,
    \end{tikzcd}
  \end{center}
  induce una sucesión exacta
  $$
  1 \rightarrow \FF(S) \overset{i_S}{\rightarrow} \GG(S) \overset{\pi_S}{\rightarrow}  \HH(S) \overset{\delta}{\rightarrow} H^1(S,\FF) \overset{i_*}{\rightarrow} H^1(S,\GG) \overset{\pi_*}{\rightarrow}  H^1(S,\HH).
      $$
\end{thm}

\begin{proof}
  En la discusión anterior ya hemos demostrado la exactitud de la sucesión hasta $\HH(S)$. La exactitud en $H^1(S,\FF)$ es clara de la construcción de la aplicación $\delta$. Por último, la demostración de la exactitud en $H^1(S,\GG)$ es totalmente análoga a la de la exactitud en $\GG(S)$.
\end{proof}

\begin{ejemplo}[Particiones diferenciables de la unidad]
  Antes de cerrar la sección, vamos a hacer un ejemplo de cálculo de un grupo de cohomología. Supongamos que $S$ es una variedad diferenciable y consideremos el haz $C^\infty_S$ de las funciones diferenciables en $S$ (con valores en $\CC$, por mantener la consistencia en la notación, aunque esto es irrelevante para el ejemplo ya que podríamos tomarlas con valores en $\RR$).
  
  Se trata de calcular $H^1(S,C^\infty_S)$. La clave para este cálculo es que en este caso existen las \emph{particiones diferenciables de la unidad}. Si $\mathfrak{U}$ es un recubrimiento por abiertos de $S$, entonces es posible tomar un conjunto de funciones diferenciables globalmente definidas $$\left\{ \theta_U : U \in \mathfrak{U} \right\}  \subset C^{\infty}_S(S)$$
  tal que
  \begin{enumerate}
    \item para cada $U\in \mathfrak{U}$, se tiene que $\theta_U(p)=0$ si $p\not \in U$, y
    \item $\sum_{U\in \mathfrak{U}} \theta_U = 1$.
  \end{enumerate}
  Este conjunto se llama una \emph{partición diferenciable de la unidad} subordinada a $\UU$.

  Supongamos entonces que tenemos un par $(\mathfrak{U},(f_{UV})_{U,V\in \UU})$ representando a una clase en $H^1(S,C^\infty_S)$. Si tomamos una partición diferenciable de la unidad subordinada a $\UU$, podemos definir, para cada $U\in \UU$,
  \begin{equation*}
    f_U = \sum_{W\in \UU} \theta_W f_{UW} \in C^\infty_S(U).
  \end{equation*}
  Ahora, para $U,V \in \UU$, en $U\cap V$ tenemos
  \begin{equation*}
    f_U - f_V = \sum_{W\in \UU} \theta_W (f_{UW} - f_{VW}) = \sum_{W\in \UU} \theta_W f_{UV} = f_{UV}.
  \end{equation*}
  Concluimos entonces que $H^1(S,C^\infty_S)=0$.

  Este argumento sirve análogamente para cualquier haz en el que podamos definir particiones de la unidad. En particular, se generaliza directamente para probar que los haces de formas diferenciales \emph{diferenciables}, que consideraremos más adelante, tienen todos cohomología nula. 
  
  Sin embargo, existen haces muy importantes que consideraremos que no admiten particiones de la unidad. Un ejemplo importante son los \emph{haces de funciones holomorfas}. Por ejemplo, si $D\subset \CC$ es un dominio, podemos considerar el haz $\OO_D$ definido de manera que si $U\subset D$ es un conjunto abierto, $\OO_D(U)$ es el conjunto de funciones holomorfas definidas en $U$. En este caso es claro que no es posible definir particiones de la unidad, ya que si una función holomorfa se anula en un abierto de $D$, debe anularse en todo $D$.
\end{ejemplo}

\section{Superficies de Riemann}

\subsection{El concepto}
\begin{defn}
  Una \emph{superficie de Riemann} es una terna $(S,\mathfrak{U},\Phi)$, donde $S$ es una superficie topológica conexa, $\mathfrak{U}$ es un recubrimiento por abiertos de $S$ y $$\Phi=\{\phi_U: U\in \mathfrak{U}\}$$ es un conjunto de aplicaciones tales que:
  \begin{enumerate}
    \item cada $\phi_U:U\rightarrow \CC$ es un homeomorfismo sobre un subconjunto abierto de $\CC$, y
    \item las funciones $\phi_U \circ \phi_V^{-1}: \phi_V(U\cap V) \rightarrow \phi_U(U\cap V)$ son holomorfas.
  \end{enumerate}

  Un par de la forma $(U,\phi_U)$, para $U\in \mathfrak{U}$, se llama un \emph{sistema de coordenadas holomorfas}.
\end{defn}

Dada una superficie de Riemann $X=(S,\mathfrak{U},\Phi)$, decimos que una función $f:X\rightarrow \CC$ es \emph{holomorfa} en un punto $p \in S$ si existe un abierto $U\in \mathfrak{U}$, con $p\in U$, tal que $f\circ \phi_U^{-1}:\phi_U(U) \rightarrow \CC$ es holomorfa en el punto $\phi_U(p) \in \CC$. Decimos que $f$ es holomorfa en un subconjunto $A\subset S$ si es holomorfa en $p$ para todo $p\in A$. Denotamos por $\OO_X(A)$ el conjunto de funciones holomorfas en $A$. Definimos el \emph{haz de estructura} $\OO_X$ de $X$, como el functor que asocia a cada abierto $U$ de $X$ el conjunto $\OO_X(U)$ y a cada inclusión $V\subset U$ el homomorfismo de restricción $\OO_X(U)\rightarrow \OO_X(V)$.

\begin{defn}
  Sean $S$ y $T$ dos espacios topológicos, $f:S\rightarrow T$ una función continua y $\mathscr{F}$ un haz en $S$. Definimos el haz imagen directa $f_*\mathscr{F}$ sobre $T$ como 
  \begin{equation*}
    f_*\mathscr{F}(U)=\mathscr{F}(f^{-1}(U)),
  \end{equation*}
  para cada $U\subset T$.
\end{defn}

\begin{defn}
  Una \emph{estructura compleja} en una superficie topológica $S$ es un subhaz $\OO_X$ del haz $C_S(-,\CC)$ de funciones continuas en $S$ con valores en $\CC$ tal que existe un recubrimiento por abiertos $\mathfrak{U}$ de $X$ y un conjunto $\Phi={\phi_U:U\in \mathfrak{U}}$ de aplicaciones tales que:
  \begin{enumerate}
    \item cada $\phi_U:U\rightarrow \CC$ es un homeomorfismo sobre un subconjunto abierto de $\CC$, y
    \item para cada $U\in \mathfrak{U}$, $\phi_{U,*}(\OO_X|_U)= \OO_{\phi_U(U)}$.
  \end{enumerate}
\end{defn}

Claramente, si $X=(S,\mathfrak{U},\Phi)$ es una superficie de Riemann, entonces el haz de estructura $\OO_X$ es una estructura compleja en $S$, ya que por definición, $f\in \OO_X|_U(V)$ si y sólo si $f\circ \phi_U^{-1} \in \OO_{\phi_U(U)}(\phi_U(V))$.

Por otra parte, si $\OO_X$ es una estructura compleja en una superficie topológica $S$, ésta tiene asociada un recubrimiento $\mathfrak{U}$ y un conjunto de aplicaciones $\Phi$, de modo que la terna $(S,\mathfrak{U},\Phi)$ es una superficie de Riemann. En efecto, como $\phi_U \circ \phi_U^{-1}=\id$, que es holomorfa, tenemos que $\phi_U \circ \phi_U^{-1} \in \OO_{\phi_U(U)}(\phi_U(U\cap V))$. Por hipótesis, $\OO_{\phi_U(U)} = \phi_{U,*}(\OO_X|_U)$, de modo que $\phi_U \in \OO_X(U\cap V)$. De nuevo por la hipótesis, $\phi_{V,*}(\OO_X|_V) = \OO_{\phi_V(V)}$, luego $\phi_U \circ \phi_V^{-1} \in \OO_{\phi_V(V)}(\phi_V(U\cap V))$.

Por tanto, podemos redefinir una superficie de Riemann en la siguiente forma:

\begin{defn}
  Una superficie de Riemann es un par $X=(S,\OO_X)$, con $S$ una superficie topológica y $\OO_X$ una estructura compleja en $S$.
\end{defn}

Por abuso de notación, salvo que se especifique lo contrario, en general usaremos $X$ tanto para denotar una superficie de Riemann, como para la superficie topológica subyacente.

\subsection{El espacio tangente}
Sea $X$ una superficie de Riemann y $p\in X$ un punto. Definimos el \emph{conjunto de gérmenes de funciones holomorfas en $p$} como $\OO_{X,p}$ la espiga del haz $\OO_X$ en el punto $p$.

\begin{defn}
  Definimos el \emph{espacio tangente holomorfo a $X$ en $p$}, denotado $\mathbf{T}_pX$, como el espacio de las derivaciones en $\OO_{X,p}$; esto es, las aplicaciones $\CC$-lineales $D:\OO_{X,p}\rightarrow \CC$ que satisfacen la \emph{regla de Leibniz}:

  \begin{equation*}
    D(fg)=D(f)g(p) + f(p) D(g),
  \end{equation*}
  para cualesquiera $f,g \in \OO_{X,p}$. 
\end{defn}

\begin{obs}
 Nótese que, si $1$ es el germen de la función constante $1$, entonces, por la regla de Leibniz, $D(1)= D(1) + D(1)$, luego $D(1)=0$ y, en general, para cualquier función constante $c$, $D(c)=0$.
\end{obs}

Sea $(U,\phi_U)$ un sistema de coordenadas holomorfas con $p\in U$. Puesto que $\phi_{U,*}(\OO_X|_U)= \OO_{\phi_U(U)}$, tenemos que la asignación $f\mapsto f\circ \phi^{-1}$ da un isomorfismo entre $\OO_{X,p}$ y $\OO_{\CC,z_0}$, para $z_0=\phi_U(p)$.

\begin{prop}
  El conjunto $\OO_{\CC,z_0}$ es isomorfo a $\CC\{z\}$, el conjunto de series de potencias con coeficientes complejos en la variable $z$ con radio de convergencia positivo.
\end{prop}

\begin{proof}
  Si $f\in \OO_{\CC,z_0}$ es un germen de función holomorfa en $z_0$, entonces existe un entorno $U\subset \CC$ de $z_0$ y una función holomorfa $f_U\in \OO_{\CC}(U)$ representando a $U$. Las funciones holomorfas son analíticas, de modo que, si $T_{z_0} f_U (z) = \sum_{n=0}^\infty a_n (z-z_0)^n$ es el desarrollo de Taylor de $f_U$ en $z_0$, existe un $r>0$ tal que, para todo $z \in \CC$, con $|z-z_0|<r$, $T_{z_0} f_U (z) = f(z)$.

  Así, a cada $f\in \OO_{\CC,z_0}$ le podemos asignar la correspondiente serie de potencias $\sum_{n=0}^\infty a_n z^n \in \CC\{z\}$. Esta asignación es inyectiva, ya que si los desarrollos en serie de dos funciones analíticas coindicen en un punto, entonces estas coinciden en un abierto. La asignación es sobreyectiva, ya que para cada serie $\sum_{n=0}^\infty a_n z^n \in \CC\{z\}$ podemos definir la función $f(z) = \sum_{n=0}^\infty a_n (z-z_0)^n$, que es holomorfa al menos en el disco de convergencia de la serie.
\end{proof}

Ahora, si $f=\sum_{n=0}^\infty a_n z^n \in \CC\{z\}$ es una serie de potencias convergente y $D$ es una derivación de $\CC\{z\}$, tenemos
\begin{equation*}
  D (f) =  a_1 D(z) + \sum_{n=2}^\infty a_n D(z^n).
\end{equation*}
Pero, para $n\geq 2$,

\begin{equation*}
  D(z^n) = D(z) z^{n-1}|_{z=0} + z|_{z=0} D(z^{n-1}) = 0.
\end{equation*}
Por tanto, $D(f) = a_1 D(z)$. Nótese que el coeficiente $a_1$ es igual a la derivada $f'(0)$. Por tanto, la derivación $f\mapsto f'(0)$ es una base del espacio de derivaciones de $\CC\{z\}$.

Más generalmente, el espacio de derivaciones de $\OO_{\CC,z_0}$ está generado por la derivación
\begin{align*}
  \partial_z|_{z_0} :\OO_{\CC,z_0} &\longrightarrow \CC \\ 
  f &\longmapsto \frac{\partial f}{\partial z}(z_0). 
  \end{align*}
  En conclusión, el espacio tangente holomorfo $\mathbf{T}_pX$ es un espacio vectorial complejo de dimensión (compleja) $1$ generado por la derivación 
\begin{align*}
  \OO_{X,p} &\longrightarrow \CC \\ 
  f &\longmapsto \frac{\partial (f\circ \phi_U^{-1})}{\partial z}(z_0). 
  \end{align*}

  Toda función holomorfa $f:D\subset \CC \rightarrow \CC$ definida en un dominio $D\subset \CC$ es en particular una aplicación diferenciable $f:\RR^2 \rightarrow \RR^2$, de modo que una estructura compleja $\OO_X$ en una superficie topológica $S$ la dota en particular de una estructura de variedad diferenciable de dimensión $2$. Como tal, en cada punto $p\in S$, la superficie $S$ tiene un espacio tangente «diferenciable», que denotamos por $T_pS$. Si $(U,\phi_U)$ es un sistema de coordenadas holomorfas con $p\in U$ y escribimos $\phi_U(p)=z_0=x_0+iy_0$, es fácil ver, por un razonamiento análogo al anterior, que $T_pX$ es un espacio vectorial real de dimensión $2$ generado por las derivaciones $\partial_x|_p$ y $\partial_y|_p$. 

  Para relacionar $T_pS$ con $\mathbf{T}_p X$, consideramos la complexificación

  \begin{equation*}
    T_pS_{\CC} = T_p S \otimes_{\RR} \CC,
  \end{equation*}
  e introducimos las derivaciones
  \begin{equation*}
    \partial_z|_p = \tfrac{1}{2}(\partial_x|_p - i \partial_y|_p), \ \ \partial_{\bar{z}}|_p = \tfrac{1}{2} (\partial_x|_p + i \partial_y|_p).
  \end{equation*}
  Esto nos da una descomposición
  \begin{equation*}
    T_p S_\CC = T_p^{1,0} X \oplus T_p^{0,1} X,   
  \end{equation*}
  donde $T_p^{1,0}X$ es el subespacio generado por $\partial_z|_p$ y $T_p^{0,1} X$ el generado por $\partial_{\bar{z}}|_p$. 

  Si consideramos ahora una aplicación diferenciable $f:X\rightarrow \CC$ de la forma
  \begin{align*}
    f\circ \phi_U^{-1} :\CC &\longrightarrow \CC\\ 
      x+iy &\longmapsto u(x,y) + i v(x,y), 
    \end{align*}
    tenemos
    \begin{align*}
      \partial_{\bar{z}}|_p f &= \frac{1}{2} \left[ \left( \frac{\partial u}{\partial x}(z) + i \frac{\partial v}{\partial x}(z) \right) + i\left( \frac{\partial u}{\partial y}(x) + i \frac{\partial v}{\partial y}(z) \right) \right] \\
 &= \frac{1}{2} \left[ \left( \frac{\partial u}{\partial x}(z) -  \frac{\partial v}{\partial y}(z) \right) + i\left( \frac{\partial u}{\partial y}(x) +  \frac{\partial v}{\partial x}(z) \right) \right].
    \end{align*}
    Ahora, si $f$ es también holomorfa en $p$, debe satisfacer las ecuaciones de Cauchy-Riemann
    \begin{align*}
      \begin{cases}
      \dfrac{\partial u}{\partial x}(z) = \dfrac{\partial v}{\partial y}(z), \\
      \dfrac{\partial u}{\partial y}(z) = -\dfrac{\partial v}{\partial x}(z).
      \end{cases}
    \end{align*}
    Por tanto, tenemos que $\partial_{\bar{z}}|_p f = 0$. Concluimos entonces que 
    \begin{equation*}
      \mathbf{T}_p X = T^{1,0}_p X = \Span(\partial_z|_p).
    \end{equation*}

    \begin{obs}
      Nótese que denotamos $T_p^{1,0}X$ y no $T_p^{1,0}S$, ya que para que la descomposición varíe de forma holomorfa con el punto es crucial la existencia de la estructura holomorfa $X$. Puede tratarse de definir una estructura compleja a partir de una descomposición de este tipo. Para ello se considera el \emph{fibrado tangente} de $S$:
      \begin{equation*}
	TS = \{(p,v) : p \in S, v\in T_p S\},
      \end{equation*}
      (este es un ejemplo de \emph{fibrado vectorial}, que estudiaremos más adelante), y se considera un endomorfismo $J:TS \rightarrow TS$ con $J^2=-\id_{TS}$. La descomposición en autoespacios permite escribir $TS= T^{1,0} S \oplus T^{0,1} S$, donde la parte $(1,0)$ corresponde al autovalor $i$ y la parte $(0,1)$ al autovalor $-i$. Una $J$ de esta forma se llama una \emph{estructura casi compleja} en $S$. El teorema de Newlander--Niremberg afirma que una estrucutra casi compleja en una superficie $S$ define una estructura compleja.
    \end{obs}

    \subsection{Formas diferenciales}
    Supongamos que $X=(S,\OO_X)$ es una superficie de Riemann y $p\in S$ es un punto. Tomemos también $(U,\phi_U)$ un sistema de coordenadas holomorfas con $p \in U$. Podemos considerar el dual del espacio tangente, $T^*_p S$, que llamamos el \emph{espacio cotangente} a $S$ en $p$. Si complexificamos también este espacio, obtenemos el espacio $T^*_pS_{\CC} = T^*_p S \otimes_\RR \CC$, que está generado por los elementos duales de la base $\{\partial_z|_p, \partial_{\bar{z}}|_p\}$, que denotamos por $\{dz_p, d\bar{z}_p\}$. 

    De esta forma, a partir del conjunto de $1$-formas diferenciales en $U$ con valores en $\RR$, que denotamos por $\Omega^1(U,\RR)$, podemos definir el conjunto de \emph{$1$-formas diferenciales en $U$ con valores en $\CC$},
    \begin{equation*}
      \Omega^1(U)=\Omega^1(U,\CC) = \Omega^1(U,\RR) \otimes \CC.
    \end{equation*}
    Igual que hicimos con el espacio tangente, podemos descomponer $T^*_p S_\CC = (T_p^{1,0}X) ^* \oplus (T_p^{0,1})^*$, de modo que obtenemos
    \begin{equation*}
      \Omega^1(U)=\Omega^{1,0}(U) \oplus \Omega^{0,1}(U).
    \end{equation*}
    Equivalentemente, lo que estamos diciendo es que cualquier $1$-forma compleja $\omega\in \Omega^1(U)$ puede escribirse como
    \begin{equation*}
      \omega= \omega_{1,0} dz + \omega_{0,1}d\bar{z},
    \end{equation*}
    con $\omega_{1,0},\omega_{0,1}:U \rightarrow \CC$ funciones diferenciables.

    Tomando potencias exteriores, podemos extender esta descomposición a formas de grado superior, de modo que obtenemos
    \begin{equation*}
      \Omega^k(U) = \bigoplus_{p+q=k} \Omega^{p,q}(U),
    \end{equation*}
    con 
    \begin{equation*}
      \Omega^{p,q}(U)= \left( \wedge^p \Omega^{1,0}(U) \right) \wedge \left( \wedge^q \Omega^{0,1}(U) \right).
    \end{equation*}
    En cualquier caso, como $S$ es una superficie (es decir, tiene dimensión $2$ como variedad diferenciable), tenemos que $\Omega^k(U)=0$ para $k>2$. Así, el único caso no trivial que nos queda por estudiar en esta situación es el de las $2$-formas,
    \begin{equation*}
      \Omega^2(U)= \Omega^{2,0}(U) \oplus \Omega^{1,1}(U) \oplus \Omega^{0,2}(U)= \Omega^{1,1}(U),
    \end{equation*}
    ya que $dz\wedge dz = 0$ y $d\bar{z} \wedge d\bar{z}= 0$. Esto es, cualquier $2$-forma $\omega \in \Omega^2(U)$ puede escribirse como
    \begin{equation*}
      \omega = \omega_{1,1} dz \wedge d\bar{z},
    \end{equation*}
    con $\omega_{1,1}:U\rightarrow \CC$ una función diferenciable.

    Esta descomposición resulta también en una escisión de la diferencial exterior $d:\Omega^k(U) \rightarrow \Omega^{k+1}(U)$. Para una $0$-forma $f\in \Omega^0(U)=C^\infty(U,\CC)$, tenemos
    \begin{equation*}
      df = \partial_ zf  dz + \partial _{\bar{z}} f d\bar{z}.
    \end{equation*}
    Podemos definir entonces los operadores
    \begin{align*}
      \partial :\Omega^0(U)&\longrightarrow \Omega^{1,0}(U)\\ 
        f &\longmapsto \partial_z f dz, 
      \end{align*}
      y
    \begin{align*}
      \delbar :\Omega^0(U)&\longrightarrow \Omega^{0,1}(U)\\ 
      f &\longmapsto \partial_{\bar{z}} f d\bar{z}, 
      \end{align*}
      de modo que
      \begin{equation*}
	df = \partial f + \delbar f.
      \end{equation*}
      Para una $1$-forma $\omega=\omega_{1,0} dz + \omega_{0,1} d\bar{z}$, tenemos
      \begin{equation*}
	d\omega = \partial_{\bar{z}} \omega_{1,0} d\bar{z} \wedge dz + \partial_z \omega_{0,1} dz \wedge d\bar{z} = (\partial_{z} \omega_{0,1}-\partial_{\bar{z}} \omega_{1,0}) dz\wedge d\bar{z}.
      \end{equation*}
      Podemos definir los operadores
      \begin{align*}
	\partial :\Omega^{1,0}(U)&\longrightarrow \Omega^{1,1}(U)\\ 
	\omega_{1,0} dz &\longmapsto \partial_{\bar{z}} \omega_{1,0}d\bar{z} \wedge dz 
	\end{align*}
	y
      \begin{align*}
	\delbar :\Omega^{0,1}(U)&\longrightarrow \Omega^{1,1}(U)\\ 
	\omega_{0,1} d\bar{z} &\longmapsto \partial_{z} \omega_{0,1} d\bar{z} \wedge dz ,
	\end{align*}
	y las proyecciones
	\begin{align*}
	  \pr_{1,0} :\Omega^1(U)&\longrightarrow \Omega^{1,0}(U)\\ 
	  \omega_{1,0}dz + \omega_{0,1} d\bar{z} &\longmapsto \omega_{1,0}dz 
	  \end{align*}
	  y
	\begin{align*}
	  \pr_{0,1} :\Omega^1(U)&\longrightarrow \Omega^{0,1}(U)\\ 
	  \omega_{1,0}dz + \omega_{0,1} d\bar{z} &\longmapsto \omega_{0,1}d\bar{z}, 
	  \end{align*}
	  de modo que
	  \begin{equation*}
	    d\omega = \delbar \pr_{1,0} \omega + \partial \pr_{0,1} \omega.
	  \end{equation*}

	  
	  Diagramáticamente, tenemos
	  
	  \begin{center}      
	    \begin{tikzcd}
	        \Omega^0(U) \arrow{r}{d} & \Omega^1(U) \arrow{r}{d} & \Omega^2(U) \arrow{r}{d} & 0 	\\ 	  & \Omega^{1,0}(U)\arrow{rd}{\bar{\partial}}\arrow[bend left]{rrd}{\partial} & & \\	  \Omega^{0,0}(U) \arrow{ru}{\partial} \arrow{rd}{\bar{\partial}} & &  \Omega^{1,1}(U) \arrow[shift left]{r}{\partial}  \arrow[shift right]{r}[anchor=north]{\bar{\partial}}& 0.\\	  & \Omega^{0,1}(U)\arrow{ru}{\partial} \arrow[bend right]{rru}{\bar{\partial}} & &       \end{tikzcd}    \end{center}

	  Nótense en particular los segmentos
\begin{center}  \begin{tikzcd}   
    
    \Omega^{0}(U) \arrow{r}{\bar{\partial}} & \Omega^{0,1}(U) \arrow{r}{\bar{\partial}} & 0, \\
    \Omega^{1,0}(U) \arrow{r}{\bar{\partial}} & \Omega^{1,1}(U) \arrow{r}{\bar{\partial}} & 0.  \end{tikzcd}\end{center}
El conjunto $\ker(\delbar:\Omega^0(U) \rightarrow \Omega^{0,1}(U))$ consiste precisamente en las funciones holomorfas en $U$. Por otra parte, podemos considerar el conjunto de las \emph{$1$-formas diferenciales holomorfas en $U$}:
\begin{equation*}  \boldsymbol{\Omega}^1(U) = \ker(\bar{\partial}: \Omega^{1,0} \rightarrow \Omega^{1,1}).\end{equation*}
Concretamente, un elemento en $\boldsymbol{\Omega}^1(U)$ es de la forma

\begin{equation*}  \omega dz,\end{equation*} con $\omega:U\rightarrow \CC$ una función holomorfa (de modo que $\partial_{\bar{z}}\omega=0$).

Ahora, tenemos el siguiente resultado:

\begin{prop}[Lema de Poincaré para el operador $\bar{\partial}$]  Sea $D\subset \CC$ un disco abierto y $\omega \in C^\infty(\bar{D})$ una función. La función  \begin{equation*}    \alpha(z) = \frac{1}{2\pi i}\int_D \frac{\omega(w)}{w-z} dw \wedge d\bar{w},  \end{equation*}  está en $C^\infty(D)$ y satisface  \begin{equation*}    \frac{\partial \alpha}{\partial \bar{z}}=\omega.  \end{equation*}\end{prop}

Como consecuencia, para cada $\omega d\bar{z} \in \Omega^{1,0}(U)$ y para cada $\omega d\bar{z}\wedge dz \in \Omega^{1,1}(U)$ podemos tomar un subconjunto  $D \subset U$ homeomorfo a un disco abierto y la función \begin{equation*}    \alpha(z) = \frac{1}{2\pi i}\int_D \frac{\omega(w)}{w-z} dw \wedge d\bar{w}  \end{equation*}  satisface $\partial_{\bar{z}} \alpha = \omega$, de modo que \begin{equation*}  \omega d\bar{z} = \delbar{\alpha} \ \ \text{ y } \ \   \omega d\bar{z}\wedge dz = \bar{\partial}(\alpha dz),  \end{equation*} en $D$. En conclusión, las aplicaciones $\delbar:\Omega^0(D)\rightarrow \Omega^{0,1}(D)$ y $\bar{\partial}:\Omega^{1,0}(D) \rightarrow \Omega^{1,1}(D)$ son sobreyectivas y las sucesiones  \begin{center}    \begin{tikzcd} 
               0 \arrow{r}& \OO_X(D) \arrow[hook]{r}& \Omega^{0}(D) \arrow{r}{\bar{\partial}} & \Omega^{0,1}(D) \arrow{r}{\bar{\partial}} & 0   \\
    0 \arrow{r}& \boldsymbol{\Omega}^1(D) \arrow[hook]{r}& \Omega^{1,0}(D) \arrow{r}{\bar{\partial}} & \Omega^{1,1}(D) \arrow{r}{\bar{\partial}} & 0    \end{tikzcd}  \end{center}  son exactas.

\begin{proof}
  Hacer la DEMO (Griffiths-Harris).
\end{proof}

Globalmente, podemos considerar los haces $\Omega^k_S$, definidos por $\Omega^k_S(U)=\Omega^k(U)$. El conjunto de secciones globales $\Omega^k(X)=\Omega^k_S(S)$ es el conjunto de las $k$-formas diferenciales en $S$ con valores en $\CC$. Análogamente, podemos globalizar las mismas descomposiciones y operadores y obtenemos los haces de $(p,q)$-formas $\Omega^{p,q}_X$ y el haz de $1$-formas holomorfas $\boldsymbol{\Omega}^1_X$.

Obtenemos también un diagrama análogo

\begin{center}	    \begin{tikzcd}
      \Omega^0_S \arrow{r}{d} & \Omega^1_S \arrow{r}{d} & \Omega^2_S \arrow{r}{d} & 0 	\\ 	  & \Omega^{1,0}_X\arrow{rd}{\bar{\partial}}\arrow[bend left]{rrd}{\partial} & & \\	  \Omega^{0,0}_X \arrow{ru}{\partial} \arrow{rd}{\bar{\partial}} & &  \Omega^{1,1}_X \arrow[shift left]{r}{\partial}  \arrow[shift right]{r}[anchor=north]{\bar{\partial}}& 0.\\	&   \Omega^{0,1}_X\arrow{ru}{\partial} \arrow[bend right]{rru}{\bar{\partial}} & &       \end{tikzcd}    \end{center}

  Teniendo en cuenta las sucesiones exactas anteriores, obtenemos las sucesiones exactas
 \begin{center}    \begin{tikzcd} 
               0 \arrow{r}& \OO_X \arrow[hook]{r}& \Omega^{0,0}_X \arrow{r}{\bar{\partial}} & \Omega^{0,1}_X \arrow{r}{\bar{\partial}} & 0   \\
    0 \arrow{r}& \boldsymbol{\Omega}^1_X \arrow[hook]{r}& \Omega^{1,0}_X \arrow{r}{\bar{\partial}} & \Omega^{1,1}_X \arrow{r}{\bar{\partial}} & 0.    \end{tikzcd}  \end{center}
En general, sin embargo, al tomar secciones globales obtenemos los complejos de cadenas
 \begin{center}    \begin{tikzcd} 
               0 \arrow{r}& \OO_X(X) \arrow[hook]{r}& \Omega^{0,0}(X) \arrow{r}{\bar{\partial}} & \Omega^{0,1}(X) \arrow{r}{\bar{\partial}} & 0   \\
    0 \arrow{r}& \boldsymbol{\Omega}^1(X) \arrow[hook]{r}& \Omega^{1,0}(X) \arrow{r}{\bar{\partial}} & \Omega^{1,1}(X) \arrow{r}{\bar{\partial}} & 0,    \end{tikzcd}  \end{center}
que no son necesariamente exactos. 

\begin{defn}
  Se definen los \emph{grupos de cohomología de Dolbeaut} como
  \begin{align*}
    H^{0,0}(X)&=\ker(\delbar:\Omega^{0,0}(X) \rightarrow \Omega^{0,1}(X)), \\
    H^{1,0}(X)&=\ker(\delbar:\Omega^{1,0}(X) \rightarrow \Omega^{1,1}(X)), \\
    H^{0,1}(X)&=\coker(\delbar:\Omega^{0,0}(X) \rightarrow \Omega^{0,1}(X)), \\
    H^{1,1}(X)&=\coker(\delbar:\Omega^{1,0}(X) \rightarrow \Omega^{1,1}(X)) .
  \end{align*}
\end{defn}

Como tomar secciones globales sí que respeta la exactitud por la izquierda, acabamos de ver que
\begin{align*}
  H^{0,0}(X) & \cong \OO_X(X), \\
  H^{1,0}(X) & \cong \boldsymbol{\Omega}^1 (X).
\end{align*}
Por otra parte, tenemos el siguiente resultado:

\begin{thm}[Dolbeaut]
  \begin{align*}
    H^{0,1}(X) &= H^1(X,\OO_X), \\
    H^{1,1}(X) &= H^1(X,\boldsymbol{\Omega}^1_X).
  \end{align*}
\end{thm}

\begin{proof}
  Hagamos la demostración para $H^{0,1}(X)$, siendo la otra completamente análoga. La clave está en que los haces $\Omega_X^{p,q}$, en tanto que consisten en formas diferenciales \emph{diferenciables}, admiten particiones diferenciables de la unidad. Por tanto, $H^1(X,\Omega^{0,0}_X)=0$. En consecuencia, si tomamos la sucesión exacta larga en cohomología asociada a la sucesión exacta corta
 \begin{center}    \begin{tikzcd} 
               0 \arrow{r}& \OO_X \arrow[hook]{r}& \Omega^{0,0}_X \arrow{r}{\bar{\partial}} & \Omega^{0,1}_X \arrow{r}{\bar{\partial}} & 0,   
     \end{tikzcd}  \end{center}
   obtenemos la sucesión exacta
   \begin{center}
     \begin{tikzcd}
       0 \rar & \OO_X(X) \rar & \Omega^{0,0}(X) \arrow{r}{\delbar} & \Omega^{0,1}(X) \rar & H^1(X,\OO_X) \rar & 0,
     \end{tikzcd}
   \end{center}
   de donde se deduce que
   \begin{equation*}
     H^1(X,\OO_X) = \coker(\delbar) = H^{0,1}(X).
   \end{equation*}
   Análogamente para $\boldsymbol{\Omega}^1_X$.
\end{proof}

\subsection{Teoría de Hodge}
\begin{defn}
  Sea $X$ una superficie de Riemann, definimos el operador \emph{laplaciano} como
  \begin{equation*}
    \Delta = 2i \delbar \partial: \Omega^{0,0} \rightarrow \Omega^{1,1}.
  \end{equation*}
\end{defn}

En coordenadas locales, tenemos
\begin{equation*}
  \Delta f = -2i \tfrac{1}{4} (\partial_x + i \partial_y)(\partial_x - i \partial_y) f (dz\wedge d\bar{z})  = - (\partial_x^2  + \partial_y^2 ) f dx \wedge dy,
\end{equation*}
ya que
\begin{equation*}
  dz \wedge d\bar{z} = (dx+idy) \wedge (dx - idy) = -idx\wedge dy + i dy \wedge dx = -idx\wedge dy.
\end{equation*}
Esto justifica el uso del nombre \emph{laplaciano} para el operador $\Delta$. 

Enunciamos ahora el primero de los teoremas de existencia que veremos en estas notas. Daremos la demostración en la sección final, dedicada a los teoremas de existencia.

\begin{thm}[Hodge]
  Sea $X$ una superficie de Riemann compacta y $\rho$ una $2$-forma en $X$. Existe una solución a la ecuación $$\Delta f=\rho$$
  si y sólo si $\int_X \rho = 0.$
  En tal caso, la solución $f$ es única salvo una constante.
\end{thm}

Este teorema tiene una serie de consecuencias sobre la cohomología de Dolbeaut de las superficies de Riemann compactas. En primer lugar, nótese que tenemos las siguientes aplicaciones definidas de forma natural:
\begin{itemize}
  \item Una aplicación $\sigma: H^{1,0}(X) \rightarrow \overline{H^{0,1}(X)}$ inducida por la aplicación $\omega \mapsto \bar{\omega}$.
  \item Una aplicación bilineal
    \begin{align*}
      B :H^{1,0}(X) \times H^{0,1}(X) &\longrightarrow \CC,
      \end{align*}
      definida por
      \begin{equation*}
	B(\alpha,[\theta]) = \int_X \alpha \wedge \theta.
      \end{equation*}
      Esta aplicación está bien definida ya que si cambiamos $\theta$ por $\theta+ \delbar f$, la integral cambia por
      \begin{equation*}
	\int_X \alpha \wedge \delbar f = \int_X -\delbar(f \alpha) + f\delbar \alpha=-\int_X \delbar(f\alpha)=0,
      \end{equation*}
      por el teorema de Stokes.

    \item Una aplicación $i:H^{1,0}(X) \rightarrow H^1(X,\CC)$ que envía cada $1$-forma holomorfa a su clase de cohomología de de Rham.

    \item Una aplicación $\nu: H^{1,1}(X) \rightarrow H^2(X,\CC)$ inducida por la inclusión
      \begin{equation*}
	\im(\delbar: \Omega^{1,0}(X) \rightarrow \Omega^{1,1}(X)) \subset \im(d:\Omega^1(X)\rightarrow \Omega^2(X)).
      \end{equation*}
\end{itemize}

      Obtenemos ahora la siguiente consecuencia del teorema de Hodge:
      \begin{corol}
	Sea $X$ una superficie de Riemann compacta.
	\begin{enumerate}
	  \item La aplicación $\sigma$ es un isomorfismo $H^{1,0}(X) \cong \overline{H^{0,1}(X)}$.
	  \item La aplicación
	    \begin{align*}
	      H^{1,0}(X) \oplus H^{0,1}(X)&\longrightarrow H^1(X,\CC)\\ 
	      (\alpha,\theta) &\longmapsto i(\alpha) + i(\sigma^{-1}(\bar{\theta})), 
	      \end{align*}
	      es un isomorfismo. El isomorfismo $H^1(X,\CC)=H^{1,0}(X) \oplus H^{0,1}(X)$ se conoce como la \emph{descomposición de Hodge}. En consecuencia, $H^{1,0}(X)$ y $H^{0,1}(X)$ son espacios vectoriales reales de dimensión finita $g$, el género de la superficie $X$.
	  \item La aplicación $B$ induce un isomorfismo  $H^{0,1}(X) \cong (H^{1,0}(X))^*$. Este resultado se conoce como la \emph{dualidad de Serre}.
	    \item La aplicación $\nu:H^{1,1}(X) \rightarrow H^2(X)$ es un isomorfismo. En particular, por ser $X$ compacta, componiendo con el isomorfismo
	      \begin{align*}
		 H^2(X)&\longrightarrow \CC \\ 
		  \omega &\longmapsto  \int_X \omega,
		\end{align*}
		tenemos que $H^{1,1}(X)\cong \CC$.
	\end{enumerate}
      \end{corol}

      \begin{proof} \ \\
1.	Para ver que $\sigma$ es sobreyectiva tomamos una clase $[\theta] \in H^{0,1}(X)$. Queremos hallar un representante $\theta'=\theta + \delbar f$ tal que $\partial \theta'=0$, ya que, en tal caso $\delbar \bar{\theta'}=0$ y $[\theta]=-\sigma( \bar{\theta'})$. Se trata por tanto de resolver la ecuación
	\begin{equation*}
	  \partial \delbar f = -\partial \theta.
	\end{equation*}
	Como $\partial \delbar = \tfrac{i}{2} \Delta$, el teorema de Hodge garantiza que podemos resolver esta ecuación ya que, por el teorema de Stokes, $\int_X \partial \theta = 0$. 

	La aplicación
	\begin{equation*}
	 \langle \alpha,\beta \rangle=  \int_X \alpha \wedge \bar{\beta},
	\end{equation*}
	con $\alpha, \beta \in H^{1,0}(X)$ es definida positiva ya que, si localmente $\alpha=adz$,
	\begin{equation*}
	  \lVert \alpha \rVert^2 = \int_X \alpha \wedge \bar{\alpha} = \int_X a \bar{a} dz \wedge d\bar{z} = \int_X  |a|^2 dz \wedge d\bar{z}.
	\end{equation*}
	En particular, si $\alpha \in H^{1,0}(X)$, entonces
	\begin{equation*}
	  B(\alpha,\sigma(\alpha)) = \lVert \alpha \rVert ^2,
	\end{equation*}
	que es distinto de $0$ si y sólo si $\alpha=0$. Esto prueba que $\sigma$ es inyectiva. \ \\
	\ \\
	2. En primer lugar, veamos que $i$ es inyectiva. Para ello, observemos que si $\alpha=df$ es una $1$-forma holomorfa, entonces $\alpha=\partial f$ por ser de tipo $(1,0)$ y $\delbar \alpha = 0$ por ser holomorfa. Esto implica que $\Delta f =0$ y, por el teorema de Hodge, $f$ es una función constante. Por tanto $\alpha = df =0$. En consecuencia, la aplicación del apartado 2 es inyectiva.

	Por otra parte, toda $1$-forma $\alpha$ se descompone en tipos como $\alpha^{1,0} + \alpha^{0,1}$. Si $d\alpha =0$, entonces $\delbar \alpha^{1,0} =0$ y $\partial \alpha^{0,1}=0$. La segunda ecuación puede reescribirse como $\delbar \overline{\alpha^{0,1}} = 0$, de modo que
	\begin{equation*}
	  [\alpha] = i(\alpha^{1,0}) + i(\sigma^{-1}(\overline{\alpha^{0,1}})).
	\end{equation*}
	Ahora, $\dim_{\RR} H^1(X,\CC) = 2g$, y como $H^{1,0}(X)$ y $H^{0,1}(X)$ tienen la misma dimensión, ésta ha de ser igual a $g$. \ \\
	\ \\
	3. Como los espacios de cohomología de Dolbeaut son de dimensión finita, la fórmula $B(\alpha, \sigma(\alpha)) = \lVert \alpha \rVert ^2$ y el hecho de que $\sigma$ es un isomorfismo implican que $B$ es un emparejamiento perfecto y que por tanto induce un isomorfismo $H^{0,1}(X) \cong (H^{1,0}(X))^*$. \ \\
	\ \\
	4. Finalmente, claramente $\nu$ es sobreyectiva, ya que toda $2$-forma en una superficie de Riemann es una $(1,1)$-forma. Podemos ver que es inyectiva de nuevo por el teorema de Hodge. En efecto, si $\omega=d\alpha$, entonces ha de existir una $f$ tal que $\omega=\Delta f$. Por tanto, $\omega = \delbar(2i\partial f)$. Es decir, que si $\nu([\omega])=0$, entonces también $[\omega]=0$.
      \end{proof}

      \section{Fibrados vectoriales}
 

\begin{thebibliography}{9}

\bibitem{Be-spanish}
\textsc{J. Bezos}, 
Estilo spanish para el sistema babel,
\url{http://www.ctan.org/tex-archive/language/spanish/babel/spanish.pdf}

\bibitem{Be-orto}
\textsc{J. Bezos}, 
Ortotipograf\'{\i}a y notaciones matem\'aticas,
\url{http://www.tex-tipografia.com/archive/ortomatem.pdf}

\bibitem{latex-imprenta}
\textsc{B. Cascales, P. Lucas, J. M. Mira, A. Pallar\'es y S. S\'anchez-Pedre\~no}, 
\textit{LaTeX, una imprenta en tus manos}, 
Aula Documental de Investigaci\'on, Madrid, 2000.

\bibitem{Do}
\textsc{M. Downes},
Short math guide for \LaTeX,
\url{ftp://ftp.ams.org/pub/tex/doc/amsmath/short-math-guide.pdf}

\bibitem{Kn}
\textsc{D. E. Knuth}, 
Mathematical typography, 
\textit{Bull. Amer. Math. Soc. (N.S.)} 
\textbf{1} (1979), 337--372. 

\bibitem{MeSl}
\textsc{A. Mertz y W. Slough},
Graphics with PGF and Ti\textit{k}Z,
\textit{The Prac\TeX\ Journal} 2007, n.\textsuperscript{o}~1.
Disponible en
\url{http://www.tug.org/pracjourn/2007-1/mertz/}

\bibitem{latex-companion}
\textsc{F. Mittelbach, M. Goossens, J. Braams, D. Carlisle y C. Rowley}, 
\textit{The \LaTeX\ Companion}, 
2.\textsuperscript{a} ed., 
Addison-Wesley, 2004. 

\bibitem{Tao-ICM} 
\textsc{T. Tao}, 
The dichotomy between structure and randomness, arithmetic progressions, and the primes, 
\textit{International Congress of Mathematicians} (Madrid, 2006), 
Vol. I, 581--608, \textit{Eur. Math. Soc.}, Zurich, 2007.

\end{thebibliography}

%---------------------------
\end{document}
%---------------------------

